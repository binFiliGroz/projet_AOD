\documentclass[a4paper, 10pt, french]{article}
% Préambule; packages qui peuvent être utiles
   \RequirePackage[T1]{fontenc}        % Ce package pourrit les pdf...
   \RequirePackage{babel,indentfirst}  % Pour les césures correctes,
                                       % et pour indenter au début de chaque paragraphe
   \RequirePackage[utf8]{inputenc}   % Pour pouvoir utiliser directement les accents
                                     % et autres caractères français
   % \RequirePackage{lmodern,tgpagella} % Police de caractères
   \textwidth 17cm \textheight 25cm \oddsidemargin -0.24cm % Définition taille de la page
   \evensidemargin -1.24cm \topskip 0cm \headheight -1.5cm % Définition des marges
   \RequirePackage{latexsym}                  % Symboles
   \RequirePackage{amsmath}                   % Symboles mathématiques
   \RequirePackage{tikz}   % Pour faire des schémas
   \RequirePackage{graphicx} % Pour inclure des images
   \RequirePackage{listings} % pour mettre des listings
% Fin Préambule; package qui peuvent être utiles

\title{Rapport de TP 4MMAOD : Génération d'ABR optimal}
\author{
GROZ Irénée 
\\FILIPPI Angelo 
\\(groupe 11) 
}

\begin{document}

\maketitle


%%%%%%%%%%%%%%%%%%%%%%%%%%%%%%%%%%%%%%%%%%%%%%
\section{Principe de notre  programme (1 point)}
Notre programme est récursif donc l'implémentation des équations de Bellman elles-mêmes récursives a été directe.
Plusieurs optimisations sont présentes :
\begin{itemize}
\item[\underline{précalcul de la somme des fréquences :}] pour éviter de faire une somme de fréquences à chaque appel de la fonction récursive nous avons stocké les fréquences sour forme cumulée afin de se ramener à une seule différence lors du calcul de la somme
\item[\underline{mémoïsation}] les profondeurs moyennes sont stockées dans une matrice pour éviter de refaire certains calculs intermédiaires un grand nombre de fois
\item[\underline{bornes de recherche :}] au lieu de faire varier l'indice de recherche \textbf{k} du minimum sur tout l'intervalle \textbf{[i, j]}, nous avons restreint la recherche à l'intervalle \textbf{[racine(i, j-1), racine(i+1, j)]}.
\end{itemize}

%%%%%%%%%%%%%%%%%%%%%%%%%%%%%%%%%%%%%%%%%%%%%%
\section{Analyse du coût théorique (2 points)}
{\em Donner ici l'analyse du coût théorique de votre programme en fonction du nombre $n$ d'éléments dans le dictionnaire.
 Pour chaque coût, donner la formule qui le caractérise (en justifiant brièvement pourquoi cette formule correspond à votre programme), 
 puis l'ordre du coût en fonction de $n$ en notation $\Theta$ de préférence, sinon $O$.}

  \subsection{Nombre  d'opérations en pire cas\,: }
    \paragraph{Justification\,: }
    {\em La justification peut être par exemple de la forme: \\ 
       "Le programme itératif contient les boucles $k_1=...$, $k_2= ...$ etc correspondant à la somme 
      $$T(n_1, n_2, c_1, c_2) = \sum_{k_1=...}^{...} ... \sum ... + \sum_{i=...}^{...} ...$$ 
      somme que nous avons calculée (ou majorée) par la technique de  ... " \\
      ou  encore\,:  \\
      "les appels récursifs du programme permettent de modéliser son coût par le système d'équations aux récurrences 
      $$T(k_1, k_2) = ...  \mbox{~avec~les~conditions~initiales~....~} $$
      Le coût indiqué est obtenu en résolvant ce système par la méthode de  .... "
    } 
  \subsection{Place mémoire requise\,: $O(n^2)$}
    \paragraph{Justification\,: } Les allocations mémoires nécessaires sont en $O(n)$ pour le tableau des fréquences cumulées et en $O(n^2)$ pour les matrices de racines et des profondeurs.

  \subsection{Nombre de défauts de cache sur le modèle CO\,: }
    \paragraph{Justification\,: }


%%%%%%%%%%%%%%%%%%%%%%%%%%%%%%%%%%%%%%%%%%%%%%
\section{Compte rendu d'expérimentation (2 points)}
  \subsection{Conditions expérimentaless}

    \subsubsection{Description synthétique de la machine\,:} 
	La machine utilisée pour les tests possède un Intel Celeron 2955u à 1.4GHz. Elle tourne sous GalliumOS 2.1 (basé sur Xubuntu). Elle possède 4Go de DDR3 à 1600MHz. \\
	La machine était monopolisée pour les tests.

    \subsubsection{Méthode utilisée pour les mesures de temps\,: } 
    Les tests ont été lancé par la commande "make test" qui appelle un script qui lance les benchmarks de manière séquentielle (benchmark1 puis benchmark 2 ...).
    La commande a été utilisé 5 fois d'affilée pour les mesures.\\
    
    Le temps est mesuré au sein du programme c via la fonction \emph{clock()} de \textit{time.h}.

  \subsection{Mesures expérimentales}

    \begin{figure}[h]
      \begin{center}
        \begin{tabular}{|l||r||r|r|r||}
          \hline
          \hline
            & n			& temps     & temps   & temps \\
            &			& min (s)   & max (s) & moyen (s)\\
          \hline
          \hline
            benchmark1 & 5 & 0.000017 & 0.000021 & 0.000019 \\
          \hline
            benchmark2 & 10 & 0.000023 & 0.000034 & 0.000027 \\
          \hline
            benchmark3 & 1000 & 0.061833 & 0.062367 & 0.062062 \\
          \hline
            benchmark4 & 2000 & 0.285534 & 0.286234 & 0.285952 \\
          \hline
            benchmark5 & 3000 & 0.673821 & 0.680874 & 0.676686 \\
          \hline
            benchmark6 & 5000 & 2.008140 & 2.030586 & 2.019177 \\
          \hline
          \hline
        \end{tabular}
        \caption{Mesures des temps minimum, maximum et moyen de 5 exécutions pour les 6 benchmarks.}
        \label{table-temps}
      \end{center}
    \end{figure}

\subsection{Analyse des résultats expérimentaux}
{\em Donner  une réponse justifiée  à la question\,: 
              les  temps mesurés correspondent ils  à votre analyse théorique (nombre d’opérations et défauts de cache) ?
}
Par une régression quadratique sur les résultats expérimentaux on vérifie que le temps moyen est en $O(n^2)$.

\end{document}
%% Fin mise au format

